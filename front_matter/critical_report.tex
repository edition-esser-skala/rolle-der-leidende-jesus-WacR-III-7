\documentclass[shorttitlesize=55,tocstyle=ref-genre]{ees}

\begin{document}

\eesTitlePage

\def\eesCommentaryAfterToe{
Bass figures appear in the following movements (bars in parentheses): 1.1, 1.3, 1.5 (1), 1.7, 1.8, 1.10~(94), 1.11, 1.12 (2, 18, 26), 1.14 (3, 6, 9), 1.16 (1, 2, 15, 17), 1.18 (11, 16, 17), 1.21, 2.1 (18, 20, 22, 59, 60, 79), 2.3 (5, 24), 2.4 (8, 16, 17, 20, 28), 2.6, 2.9 (3, 4, 6, 8, 11, 21), 2.10 (1–3, 14, 17), 2.11 (5), 2.13~(1–3, 8, 18), 2.15 (4–9, 18), 2.18 (14), and 2.20 (4, 13). The remaining bass figures were added by the editor.
}

\eesCriticalReport{
  1.3  & 20  & A     & 1st \quarterNote\ in \B1: g′4 \\
  1.6  & 109 & org   & bar in \B1: G2–\crotchetRest \\
  1.10 & 52  & T     & grace note missing in \B1 \\
  1.11 & 25f & T     & bars in \B1: c′2–g2–a2–f′2 \\
  1.21 & –   & –     & instruments missing in \B1 \\
  2.1  & 67  & org   & 1st \quarterNote\ in \B1: A4 \\
  2.5  & 44  & cor 1 & bar in \B1: e′2–\crotchetRest \\
  2.7  & –   & –     & Flute staves are empty in \B1; they might play unison with vl 1, 2 (possibly only if the latter instruments play forte). \\
}

\eesToc{
\part{erstertheil}

\begin{movement}{omeineseel}
  \voice[Coro]
  O, meine Seel, ermuntre dich,\\
  daß ganz auf Jeſu Leiden ſich\\
  voll Andacht Herz und Auge lenkt,\\
  und Himmelswonne dann dich tränkt.

  Anbetend folge nun ihm nach,\\
  von Hohn zu Hohn, von Schmach zu Schmach,\\
  bis zu dem Kreuz, daran er ſtirbt\\
  und Sündern Gottes Huld erwirbt.

  Dann ſage: Jeſu, du mein Heil,\\
  ich bin dein Eigenthum und Theil,\\
  wie du mein biſt, ſo bleib ich dein,\\
  und will mich ganz dir ewig weyhn.
\end{movement}

\begin{movement}{derherrneiget}
  \voice[Coro]
  Der Herr neiget den Himmel und fähret herab.\\
  Dunkel iſt unter ſeinen Füßen,\\
  und in finſtern Wolken iſt er verborgen.\\
  (\bibleverse{Ps}(18:10,12))
\end{movement}

\begin{movement}{jerusalem}
  \voice[Alto]
  Jeruſalem! Jeruſalem!\\
  Welch eine Nacht ſenkt über dich\\
  vom Himmel ſich hernieder!\\
  Kommt Gott, der zum Gericht erwacht,\\
  und fodert der Propheten Blut\\
  von deinen Händen wieder?\\
  Steigt er zum Weltgericht herab von ſeinem Thron,\\
  der Oelberg erbebet, es rauſchet der Kidron,\\
  von fern her rauſcht der Donnerton.\\
  Wo iſt der größte der Propheten, dein Jeſus?\\
  Fleh ihn an, dich zu verteten.\\
  Wo find ich ihn, den meine Seele liebt?\\
  Hör ich ihn dort betrübt\\
  aus jener Mitternacht her klagen?\\
  Er iſts.\\
  Mit welchen dunkeln Plagen\\
  hat ihn ſein Gott geſchlagen!
\end{movement}

\begin{movement}{gerechter}
  \voice[Alto]
  Gerechter! Welche Qual erſchrecket\\
  den eingebeugten Heldenmuth!\\
  Du biſt ſtatt Schweiß mit Blut bedecket,\\
  und Thränen fließen in dein Blut.\\
  Ach, weint der Heiligſte ſelbſt Zähren,\\
  faßt Schrecken ihn und banger Schmerz,\\
  wer wird mir Sünder Troſt gewähren,\\
  wer gießet Ruh in dieſes Herz?\\
  Wie ſoll ich vor dem Richter ſtehn,\\
  vor dem mein Jeſus will vergehn?
\end{movement}

\begin{movement}{labyrinth}
  \voice[Basso]
  O Labyrinth! der der Natur gebot,\\
  vor dem der Sturmwind ſchwieg,\\
  der dir, o Tod, allmächtig, ſeinen Raub entrißen,\\
  liegt hier, von Finſterniſſen\\
  umſchattet und von Angſt gedrängt,\\
  und leidet, was die Seele nicht gedenkt,\\
  und jammert Todestöne.\\
  Ach! wer erhellt mir dieſe dunkle Sonne?\\
  Mein Jeſus, meiner Seelen Freund,\\
  was iſt es, daß dein Auge weint?\\
  Heb aus dem Staube dich.\\
  Ein Blick von dir, nur einer, lehre mich,\\
  dies Todesjammern, dies Zagen\\
  verſtehn und zu ertragen.\\
  Er höret mich: ſanft iſt ſein Blick\\
  und ruft das Leben mir zurück.\\
  Sein Auge ſpricht: Die Arbeit dieſer Nacht\\
  hat mir der Menſchen Schuld gemacht.
\end{movement}

\begin{movement}{singtihr}
  \voice[Coro]
  Singt, ihr Himmel, Gott iſt Liebe,\\
  wunderbarlich iſt ſein Rath.\\
  Sing, o Erde, Gott iſt Liebe,\\
  bey der Thaten größten That.\\
  Ihr Geſchlechter der Erlöſten,\\
  dem, der leidet, uns zu tröſten,\\
  laßt uns ewig dankbar ſeyn.\\
  Jedes Herz, das ihn verkennet,\\
  nicht für ihn und Tugend brennet,\\
  faß einſt ewig dieſe Pein.
\end{movement}

\begin{movement}{liebedie}
  \voice[Coro]
  Liebe, die du mich zum Bilde\\
  deiner Gottheit haſt gemacht,\\
  Liebe, die du mich ſo milde\\
  nach dem Fall mit Heil bedacht,\\
  Liebe, dir ergeb ich mich,\\
  dein zu bleiben ewiglich.
\end{movement}

\begin{movement}{nochherrscht}
  \voice[Tenore]
  Noch herrſcht um ihn ein ſchauervolles Schweigen.\\
  Kein Laut ertönt.\\
  Wo ſeid ihr, Zeugen der Wunder, die er that?\\
  Wo ſeid ihr? Seid ihr entflohn?\\
  Vergaß ein jeder ſchon den Schwur,\\
  ſein Leben hin für den Göttlichen zu geben?\\
  Wie hatt’ er euch ſo lieb!\\
  Itzt wendet ſich der Liebende\\
  und ſuchet, Simon, dich, und euch, Zebedäiden.\\
  Mit dieſem Troſt, nur euch zu ſehn, zufrieden,\\
  naht er ſich euch.\\
  Sie ſchlummern, und er ſpricht:\\
  Ach Simon, du vermagſt es nicht,\\
  auch eine Stunde nur mit mir zu wachen!\\
  Zwar willig iſt der Geiſt,\\
  doch drückt den Schwachen das Fleiſch zur Erd herab!\\
  Ach, wacht und betet, daß ihr ſingt\\
  und nicht der Anfechtung erliegt.
\end{movement}

\begin{movement}{wachetstehet}
  \voice[Coro]
  Wachet, ſtehet im Glauben,\\
  ſeid männlich und ſeid ſtark.\\
  (\bibleverse{ICor}(16:13))
\end{movement}

\begin{movement}{herrermuntre}
  \voice[Tenore]
  Herr, ermuntre du uns Schwachen,\\
  daß wir wachen,\\
  daß der Geiſt ſich zu dir ſchwingt,\\
  kämpft und ringt.\\
  Zeig unſrem Glauben jene Höhe,\\
  wo wir gekrönt einſt vor dir ſtehn,\\
  wenn wir hier wachend dich gefunden,\\
  dir nachgekämpft und überwunden,\\
  und wo der Engel Lied den ſingt,\\
  der wacht und ringt.
\end{movement}

\begin{movement}{wieist}
  \voice[Coro]
  Wie iſt der Menſch ſo ſchwach!\\
  So viel ſein Muth verſprach,\\
  doch liegt er da im Staube.\\
  Erloſchen iſt ſein Glaube,\\
  ſein kühner Muth gedämpfet.\\
  Ach, Chriſtus wacht und kämpfet.
\end{movement}

\begin{movement}{docherverlaesst}
  \voice[Basso]
  Doch er verläßt die Schlummernden\\
  und eilt aufs neu, in das Gericht zu gehn,\\
  denn immer noch, ich ſtaune,\\
  tönt ihm in ſeinem Ohr\\
  des Weltgerichts Poſaune.\\
  Noch immer richtet der Richter der Welt den,\\
  der als Mittler zwiſchen Gott\\
  und das Geſchlecht der Menſchen ſich ſtellt.\\
  Empfindungen, ſtark wie der Tod, erſchüttern ihn.\\
  Die Erd erbebet wieder,\\
  die Nacht hängt ſchrecklicher vom Oelberg nieder.\\
  Und du, der du in Gottes Unſchuld prangſt,\\
  ringſt mit des ewgen Todes Angſt.\\
  Doch ſchau, ein Blitz zertheilet die Nacht,\\
  ein Seraph eilet herab und ſingt ein Lied\\
  von deinem Vater dir,\\
  der huldvoll auf dich ſieht,\\
  und ſtärket dich.
\end{movement}

\begin{movement}{werdurchschaut}
  \voice[Basso]
  Wer durchſchaut, wie wunderbar\\
  Gott iſt in ſeinen Werken!\\
  Ach! ein Engel muß ſogar\\
  den Herrn der Welten ſtärken.\\
  Wenn ich einſt von hinnen ſcheide,\\
  ſingen Engel mir zur Ruh.\\
  Nun eilſt du zu unſrer Freude\\
  Gottes Vaterarmen zu.
\end{movement}

\begin{movement}{gestaerkterhebt}
  \voice[Alto]
  Geſtärkt erhebt mein Jeſus ſich\\
  und geht der Schaar entgegen,\\
  die ihn voll Mordluſt ſucht,\\
  durch dich, Iſcharioth geführt.\\
  Bekümmert ſeiner Freunde wegen\\
  ſpricht er: Wen ſuchet ihr?\\
  und Allmacht iſt ſein Blick.\\
  Schnell ſtürzt die Schaar zurück\\
  und ſinkt und liegt,\\
  wie auf dem Schlachtfeld Todte liegen.\\
  Doch die Betäubung weicht,\\
  ſie ſchauen voll Vergnügen\\
  bey ihrer Fackeln Schein\\
  nach dem Verräther hin,\\
  der tritt zu ihm und küßet ihn.\\
  Und Jeſus blickt voll Mitleid,\\
  voller Ruh auf ihn, und ſagt:\\
  Juda, verrätheſt du des Menſchen Sohn\\
  mit einem Kuß?\\
  So ſanft ſucht er ihn,\\
  an des Abgrunds Stufen,\\
  zur Reu zurück zu rufen.
\end{movement}

\begin{movement}{gottmitblicken}
  \voice[Alto]
  Gott, mit Blicken deiner Gnade,\\
  hilf, daß ich vom Laſterpfade\\
  bald den Fuß zurücke zieh,\\
  meines Vaters Stimme höre,\\
  wieder reuend zu ihm kehre\\
  und des Abgrunds Tiefe flieh.\\
  Dann nimm den Reuenden\\
  mit Vaterblicken an.
\end{movement}

\begin{movement}{siebinden}
  \voice[Tenore]
  Sie binden ihn; er reicht der Schaar\\
  die Hände willig dar,\\
  indeß die kleine Zahl der Freunde ſich zerſtreuet.\\
  Mit Höllenfreuden freuet ſich Kaiphas,\\
  ſetzt auf den Richtſtuhl ſich,\\
  und richtet dich, der du voll Ruh\\
  den Blick zum Himmel lenkeſt,\\
  nur Gott, die Ewigkeit und die Erlöſten denkeſt.\\
  Der Heiligſte ſteht in der Sünder Gericht,\\
  hört die erkauften Zeugen nicht,\\
  der Läſterung, des Spottes Stimme nicht,\\
  und ſchweigt.
\end{movement}

\clearpage
\begin{movement}{lammdas}
  \voice[Coro]
  Lamm, das von verruchten Zungen\\
  frech verhöhnet dennoch ſchwieg!\\
  Stiller Muth bey Läſterungen,\\
  welch ein edelmüthger Sieg!\\
  Muß ich gleichen Grimm empfinden,\\
  lehre mich gelaßen ſeyn,\\
  und will ſich mein Zorn entzünden,\\
  flöß mir deine Sanftmuth ein.
\end{movement}

\begin{movement}{dochkaiphas}
  \voice[Soprano]
  Doch Kaiphas, ergrimmt durch dieſes Schweigen,\\
  reißt wütend ſich hervor,\\
  und ſchon glüht ihm die Wang:\\
  Schweigſt du zu dem, was dieſe zeugen? ruft er:\\
  Sprich, bey dem Ewigen beſchwör ich dich,\\
  ſprich, biſt du Gottes Sohn?\\
  Und Jeſus würdigt ihn, zu ſagen: Ich bins.\\
  Von dieſen Tagen, von nun an wirds geſchehn,\\
  daß ihr des Menſchen Sohn zur Rechten Gottes ſehn\\
  und kommend in den Wolken werdet ſehen,\\
  wenn er daher wird zum Gerichte gehen.
\end{movement}

\begin{movement}{meinistdie}
  \voice[Soprano]
  Mein iſt die Unſterblichkeit,\\
  die Unſterblichkeit iſt mein.\\
  Jauchze deinem Leben, Seele,\\
  Gott wird Ewigkeit dir zur Dauer geben.\\
  Wenn euch wird das nahe Grab erſchrecken,\\
  Todesbläße eure Wangen decken,\\
  wenn einſt dieſe Hütte ſinkt,\\
  ſchaut nach jenen Wolken dann, ihr Frommen,\\
  auf denſelben wird der Richter kommen,\\
  wenn er euch ins Leben winkt.
\end{movement}

\begin{movement}{christushatdem}
  \voice[Coro]
  Chriſtus hat dem Tode die Macht genommen\\
  und das Leben und ein unvergängliches Weſen\\
  ans Licht gebracht.\\
  (\bibleverse{IITim}(1:10))
\end{movement}

\begin{movement}{wenndort}
  \voice[Coro]
  Wenn dort, Herr Jeſu, wird vor deinem Throne\\
  auf meinem Haupte ſtehn die Ehrenkrone,\\
  da will ich dir, wenn alles wird wohlklingen,\\
  Lob und Dank ſingen.
\end{movement}

\part{zweytertheil}

\begin{movement}{weristder}
  \voice[Alto]
  Wer iſt der Mann, der unter jenen Knechten\\
  der Grauſamkeit entſchloßen ſteht,\\
  mit ihnen eifrig ſcheint zu rechten?\\
  Sie rufen: Seht! auch dieſer war ein Galiläer,\\
  wir ſahn ihn bey dem Nazaräer.\\
  Biſt du es, Simon?\\
  Ach, du wirſt ihn nicht verkennen, den Göttlichen;\\
  nein, ſo unedel biſt du nicht!\\
  O weh! er ſchwört und ſpricht:\\
  Ich kenne dieſen Menſchen nicht.\\
  Und Jeſus blickt ihn an,\\
  voll Ruh, voll Ernſt und Schmerzen,\\
  ein Dolch dringt mit dem Blick zu Petri Herzen.\\
  Er wendet ſich, und geht hinaus,\\
  und weinet bitterlich und klagt:\\
  Mein Freund, mein Freund!\\
  Ach, was that ich?\\
  Geliebt, gewarnt von dir,\\
  verläugnet dich dein Simon.\\
  Tödtend drang ſein Blick in mein Gebein.\\
  Ich fühl, ich fühle Todespein,\\
  du Göttlicher, wirſt nun mich auch nicht kennen,\\
  vor deinem Vater nicht, vor Engeln mich nicht nennen.\\
  Ja, nenne mich nur nicht, ich habs verdient,\\
  verſtoß mich im Gericht!\\
  Rauſcht, ihr Schrecken dieſer Nacht,\\
  rauſchet mir Tod und Verderben.\\
  Fluch hab ich auf mein Haupt gebracht,\\
  ach, könnt ich ſterben!\\
  Mein Vater, dieſes Herz erbebt,\\
  dies Auge weint, erbarm, o Vater,\\
  erbarm dich meiner.\\
  Viel ſündigen an ihm,\\
  der Reue Pfeil fühl keiner wie ich,\\
  er gräbet Tod mir ein!\\
  Ach laß mich, eh er ſtirbt,\\
  laß mich ihn ſehen,\\
  von ihm Verzeihung zu erflehen.\\
  Dann, wenn er ſterbend mir verzeiht,\\
  dann ſoll, ſo lang der Herr zu leben mir gebeut,\\
  vor allen Menſchen dieſer Mund ihn nennen,\\
  ihn für den theuren, beſten, göttlichſten erkennen,\\
  dann wein ich auf ſein Grab.\\
  So jammert er, und fühlt der heißen Reue Pein,\\
  Gott, Mittler, ach, erbarme du dich ſein.
\end{movement}

\begin{movement}{gottdudonnerst}
  \voice[Basso]
  Gott, du donnerſt zu den Sündern\\
  deinen Fluch vom Richterſtuhl\\
  bis hinab zum Feuerpfuhl.\\
  Ach, wenn meine Zähren fließen,\\
  wenn die Reu mein Herz zerrißen,\\
  wenn das ſtrafende Gewiſſen\\
  auf euch gießet Höllenpein,\\
  ach, Herr, höre dann mein Schreyn,\\
  gib mir Troſt, die Angſt zu lindern.
\end{movement}

\begin{movement}{achseele}
  \voice[Coro]
  Ach Seele, ſchau um welchen Preis\\
  dein Heiland dich erkaufet.\\
  Für dich rang er im Todesſchweiß,\\
  für dich im Blut getaufet.\\
  Ach Seele, ſorge, daß dich nie\\
  die Sünd in ihre Netze zieh,\\
  o Menſchenfurcht erſchüttre.\\
  Reizt dich die Welt, ach höre nicht,\\
  ſchau hin ins furchtbare Gericht,\\
  das Jeſum traf, und zittre.
\end{movement}

\begin{movement}{dertagbricht}
  \voice[Basso]
  Der Tag bricht an, der feſtliche,\\
  der große Tag, geſendet von der Liebe.\\
  Entflammet von dem Triebe der Rach und Wuth,\\
  verſammelte der Prieſter Haufe ſich.\\
  Erfüllet mir Verderben rief er:\\
  Auf Golgatha, am Kreuze ſoll er ſterben.\\
  Triumphvoll reißen ſie ihn zu Pilatus Richtſtuhl fort.\\
  Den feigen Römer ſchreckt der Kläger Wort\\
  und ihr Geſchrey, davon das Richthaus bebet:\\
  Wenn durch dein Mitleid dieſer Jeſus lebet,\\
  biſt du des Kaiſers Feind.\\
  Ihm ſinkt der Muth, doch ruft er noch:\\
  Ich bin unſchuldig an dem Blut des Frommen.\\
  Da ertönt das Schreckenswort der Sünder:\\
  Über uns komme ſein Blut und über unſre Kinder.
\end{movement}

\clearpage
\begin{movement}{todesworte}
  \voice[Coro]
  Juda, Todesworte ſprichſt du aus,\\
  erbebſt du nicht, Juda!\\
  Horch, dir jauchzt des Abgrunds Pforte,\\
  Engel wenden ihr Geſicht,\\
  ſie verlaßen dich und fliehn.\\
  Weh dir Iſrael, Rom wird dein Wort vollziehn.
\end{movement}

\begin{movement}{undjesus}
  \voice[Tenore]
  Und Jeſus wird gegeißelt.\\
  Nun träget er ſein Kreuz,\\
  mit ihm der Menſchen Sünden.\\
  Die weichern Töchter Zions folgen\\
  und empfinden ſein Leiden,\\
  und ein Zährenſtrom fließt\\
  von der Frommen Angeſicht.\\
  Er ſieht ſich tröſtend um, und ſpricht:\\
  Ihr Töchter Zions, weinet nicht.
\end{movement}

\begin{movement}{weintnicht}
  \voice[Tenore]
  Weint nicht, ſagt der Menſchenfreund\\
  Zions Töchtern, die ihn klagen.\\
  Weint nicht, ſagt der Menſchenfreund\\
  uns in kummervollen Tagen.\\
  Der ſchmale Weg zur Herrlichkeit\\
  iſt mit Blumen nicht beſtreut.\\
  Steil iſt der Pfad und rauh die Bahn,\\
  nur mühſam klimmen wir hinan,\\
  Wege, die den Sinn erfreun,\\
  laden zum Entſchlummern ein.
\end{movement}

\begin{movement}{wirmuessen}
  \voice[Coro]
  Wir müßen durch viel Trübſal\\
  in das Reich Gottes eingehen.\\
  (\bibleverse{Acts}(14:22))
\end{movement}

\begin{movement}{dastehstdu}
  \voice[Alto]
  Da ſtehſt du, Golgatha, ein Altar!\\
  Auch das Opfer iſt ſchon da,\\
  das du, Weltrichter, dir erſehen.\\
  Und Jeſus naht ſich, in den Opfertod zu gehen.\\
  Er ſchwankt den Hügel matt hinan,\\
  und ſie kreuzigen ihn.\\
  Wirf unter dieſes Kreuz, mein Geiſt,\\
  dich gläubig hin, umfaß es, hier iſt das Opfer,\\
  deſſen Blut dir zur Erlöſung fließt.\\
  Ihr Väter Iſraels, ihr höret den,\\
  der nach eurem Heil voll heißer Huld ſich ſehnet;\\
  ihr Kreuziger! ihr ſpottet ihn ins Angeſicht.\\
  Er wendet im Gebet zu ſeinem Vater ſich:\\
  Erbarme, Vater, ihrer dich,\\
  ſie wiſſen, was ſie thun, jetzt nicht.
\end{movement}

\begin{movement}{selbstderfeinde}
  \voice[Coro]
  Selbſt der Feinde Heil zu ſuchen,\\
  reiz auch uns dies Beyſpiel an.\\
  Ja wir wollen, wenn ſie fluchen,\\
  mit Gebet zu Gott uns nahn.\\
  Ihre Schuld wollſt du erlaßen,\\
  das, o Gott, iſt unſer Flehn,\\
  daß einſt ſelber, die uns haſſen,\\
  dort mit uns dein Antlitz ſehn.
\end{movement}

\begin{movement}{dermitgekreuzigte}
  \voice[Basso]
  Der Mitgekreuzigte zu ſeiner linken Hand\\
  verhöhnt ihn auch im Sterben,\\
  doch der zur Rechten, hingeriſſen zum Verderben,\\
  ſonſt edler, ſtrafet ihn und kehrt zu Jeſus ſich:\\
  Herr, kommſt du in dein Reich,\\
  ſpricht er, ſo denk an mich.\\
  Ihr Sünder, betet an; im letzten Augenblicke\\
  führt Jeſus einen Geiſt zu Gott zurücke.\\
  Ja, ſagt er, ich gedenke dein,\\
  heut wirſt du noch mit mir im Paradieſe ſeyn.
\end{movement}

\begin{movement}{theureswort}
  \voice[Soprano]
  Theures Wort aus Jeſu Munde,\\
  in der letzten Todesſtunde\\
  tröſteſt du mein brechend Herz.

  \voice[Alto]
  Theures Wort des ewgen Lebens,\\
  ach, nun ängſtet einſt vergebens\\
  meinen Geiſt der Trennung Schmerz.

  \voice[both]
  Wenn der Geiſt nun ſcheiden ſoll,\\
  macht ihn dieſes hoffnungsvoll.

  \voice[Soprano]
  An dem Tage, da ich ſterbe,\\
  ſoll ich, Jeſus, ſchon dein Erbe\\
  in dem beßern Leben ſeyn.

  \voice[Alto]
  Los von aller Furcht und Plage\\
  geh ich an demſelben Tage,\\
  Herr, in deinen Himmel ein.

  \voice[both]
  Jauchzt, Erlöſte, denn das Grab\\
  ſtürzt nie euer Leben ab.
\end{movement}

\begin{movement}{undgott}
  \voice[Soprano]
  Und Gott gebietet dem letzten Schmerz,\\
  der in des Helden Seele wütet.\\
  Ach, wer vermag die Zahl der Leiden\\
  ganz zu faſſen, die auf ihn ſtrömt?\\
  Mein Gott, mein Gott,\\
  ach, warum haſt du mich verlaßen?\\
  ſo wütet er.\\
  Nun kommt, von Gott geſandt, der Tod.\\
  Er betet: Ich vollende nun ganz das Werk,\\
  das einer Welt das Heil erwirbt,\\
  ruft: Vater, meinen Geiſt geb ich in deine Hände;\\
  und neigt ſein Haupt und ſtirbt.
\end{movement}

\begin{movement}{erbarmedich}
  \voice[Tenore I, II]
  Erbarme dich, Gott, über mich,\\
  Jeſus ſchließt ſein Leben,\\
  er ſtirbt der Verſöhnung Tod,\\
  Heil der Welt zu geben.

  Erbarme dich, Gott, über mich,\\
  ſieh mich Sünder beben,\\
  laß durch des Verſöhnungs Tod,\\
  Herr, mich wieder leben.
\end{movement}

\begin{movement}{weintnichtihr}
  \voice[Basso]
  Weint nicht, ihr Freunde Jeſu!\\
  Seht, des Richters Angeſicht ſpricht Gnade.\\
  Schauet auf, es wendet der Engelchor ſich weg, und ſingt:\\
  Es iſt vollendet, das Opfer!\\
  Gottes Ruh ſtrömt nun vom Kreuz den Sündern zu.\\
  Die Erde bebt, die Felſen ſpringen,\\
  die Gräber thun ſich auf,\\
  die Todten dringen ans Licht,\\
  des Tempels Vorhang reißt.\\
  Es ſchauet der erſtaunte Geiſt\\
  mit freyem Blick den Gnadenthron,\\
  durch Höh und Tiefen hallt der Engel Jubelton:\\
  Es iſt vollbracht!
\end{movement}

\begin{movement}{ichdanke}
  \voice[Coro]
  Ich danke dir von Herzen,\\
  o Jeſu, liebſter Freund,\\
  für deine Todesſchmertzen,\\
  wie gut haſt du’s gemeint.\\
  Ach gib, daß ich mich halte\\
  zu dir und deiner Treu,\\
  und wenn ich einſt erkalte,\\
  in dir mein Ende ſey.
\end{movement}

\begin{movement}{otod}
  \voice[Basso]
  O Tod und Grab, wo iſt dein Sieg?\\
  Nun triumphiert das Leben.

  \voice[Coro]
  Gott ſey gedankt, der uns den Sieg\\
  durch Jeſum hat gegeben.
\end{movement}

\begin{movement}{otodwoist}
  \voice[Coro]
  O Tod, wo iſt dein Stachel nun,\\
  wo iſt dein Sieg, o Hölle?\\
  Auf uns wird Gottes Frieden ruhn\\
  an dieſer Lebensſchwelle.\\
  Er, Jeſus Chriſtus, ſieht ins Grab,\\
  wir ſinken ſo wie er hinab,\\
  er ging zu Gott, wir folgen.
\end{movement}

\begin{movement}{otodc}
  \voice[Tenore]
  O Tod und Grab, wo iſt dein Sieg?\\
  Nun triumphiert das Leben.

  \voice[Coro]
  Gott ſey gedankt, der uns den Sieg\\
  durch Jeſum hat gegeben.
\end{movement}

\begin{movement}{freueteuch}
  \voice[Coro]
  Freuet euch, erlößte Seelen,\\
  fühlt nun die Unſterblichkeit.\\
  Wer kann eure Freuden zählen\\
  durch den Raum der Ewigkeit?\\
  Sagt dem Mittler dafür Dank,\\
  er ſey euer Lobgeſang,\\
  bis ihr mit des Himmels Chören\\
  ihn verkläret werdet ehren.
\end{movement}
}

\eesScore

\end{document}
