\documentclass[parskip=full]{scrreprt}

\RedeclareSectionCommand[pagestyle=plain,afterskip=10pt plus 1pt]{chapter}
\setkomafont{chapter}{\mdseries\headingfont\fontsize{40}{40}\selectfont\color{black!80}}
\setkomafont{pageheadfoot}{\normalsize}

\def\pnumbox#1{#1\hspace*{8cm}}
\def\gobble#1{}
\DeclareTOCStyleEntry[
  indent=0pt,
  beforeskip=0pt,
  entryformat=\itshape,
  entrynumberformat=\textcolor{oldred},
  numwidth=2em,
  linefill=\hfill,
  pagenumberbox=\pnumbox,
  pagenumberformat=\itshape
]{tocline}{part}

% use commented values if there are no parts
\DeclareTOCStyleEntry[
  indent=0pt,
  beforeskip=0pt,
  entrynumberformat=\textcolor{oldred},
  numwidth=2em,
  linefill=\hfill,
  pagenumberbox=\pnumbox
]{tocline}{section}

\DeclareTOCStyleEntry[
  indent=0pt,
  beforeskip=-\parskip,
  entrynumberformat=\gobble,
  entryformat=\ltseries,
  numwidth=2em,
  linefill=\hfill,
  pagenumberbox=\pnumbox,
  pagenumberformat=\ltseries
]{tocline}{subsection}

\usepackage{polyglossia}
\setdefaultlanguage{english}

\usepackage{fontspec}

\setmainfont{Source Sans Pro}[
  ItalicFont = Source Sans Pro Italic,
  BoldFont = Source Sans Pro Bold,
  BoldItalicFont = Source Sans Pro Bold Italic,
  FontFace = {lt}{n}{Source Sans Pro Light},
  FontFace = {lt}{it}{Source Sans Pro Light Italic},
  FontFace = {sb}{n}{Source Sans Pro Semibold},
  FontFace = {sb}{it}{Source Sans Pro Semibold Italic},
  Numbers = Proportional,
  Ligatures = Common
]
\DeclareRobustCommand{\ltseries}{\fontseries{lt}\selectfont}
\DeclareRobustCommand{\sbseries}{\fontseries{sb}\selectfont}
\DeclareTextFontCommand{\textlt}{\ltseries}
\DeclareTextFontCommand{\textsb}{\sbseries}

\newfontfamily\headingfont{Fredericka the Great}

\usepackage[pass]{geometry}
\newgeometry{twoside,inner=20mm,outer=40mm,top=20mm,bottom=40mm}

\usepackage{url}
\urlstyle{same}

\usepackage{microtype}
\microtypesetup{verbose=silent}

\usepackage{booktabs,array,longtable}
\newcolumntype{L}[1]{>{\raggedright\let\newline\\\arraybackslash\hspace{0pt}}p{#1}}

\usepackage{graphicx}

\usepackage{xcolor}
\definecolor{oldred}{rgb}{.8313,0,0}

\usepackage{pdfpages}

\usepackage{scrlayer-scrpage}
\pagestyle{scrheadings}
\clearscrheadfoot
\cfoot[\thepage]{\thepage}
\pagenumbering{roman}

\usepackage{enumitem}
\setlist[description]{%
	labelindent=2em,%
	labelwidth=6em,%
	leftmargin=8em,%
	labelsep=0pt,
	first=\ltseries,%
	font=\normalfont\itshape\ltseries%
}
\def\lyrefitem#1{\item[\lyref{#1}]}


\makeatletter

\def\@firstofthree#1#2#3{#1}
\def\@secondofthree#1#2#3{#2}
\def\@thirdofthree#1#2#3{#3}
\def\@firstoffour#1#2#3#4{#1}
\def\@secondoffour#1#2#3#4{#2}
\def\@thirdoffour#1#2#3#4{#3}
\def\@fourthoffour#1#2#3#4{#4}
\def\Dotfill{\leavevmode\cleaders\hb@xt@ .75em{\hss .\hss }\hfill \kern \z@}

\def\lyrefnumber#1{\expandafter\@setref\csname r@#1\endcsname\@firstofthree{#1}}
\def\lyreftitle#1{\expandafter\@setref\csname r@#1\endcsname\@secondofthree{#1}}
\def\lyrefpage#1{\expandafter\@setref\csname r@#1\endcsname\@thirdofthree{#1}}

\def\lyrefgenrenumber#1{\expandafter\@setref\csname r@#1\endcsname\@firstoffour{#1}}
\def\lyrefgenregenre#1{\expandafter\@setref\csname r@#1\endcsname\@secondoffour{#1}}
\def\lyrefgenretitle#1{\expandafter\@setref\csname r@#1\endcsname\@thirdoffour{#1}}
\def\lyrefgenrepage#1{\expandafter\@setref\csname r@#1\endcsname\@fourthoffour{#1}}

\def\lyrefpart#1{%
	\begingroup%
	\makebox[0pt][l]{\sbseries\color{oldred}\lyrefnumber{#1}}\hspace*{2em}%
	\makebox[0pt][l]{\sbseries\lyreftitle{#1}}\hspace*{6em}%
	\hfill\sbseries\lyrefpage{#1}%
	\endgroup%
}
\def\lyrefsection#1{%
	\begingroup%
	\makebox[0pt][l]{\color{oldred}\lyrefgenrenumber{#1}}\hspace*{2em}%
	\makebox[0pt][l]{\ltseries\lyrefgenregenre{#1}}\hspace*{6em}%
	\lyrefgenretitle{#1}\Dotfill\lyrefgenrepage{#1}%
	\endgroup%
}
\InputIfFileExists{../out/lilypond.ref}{}{\InputIfFileExists{../lilypond.ref}{}{}}


\newcommand\fancytitlehead{
	\headingfont%
	\fontsize{80}{80}\selectfont\textcolor{black!80}{\@ifundefined{@shortname}{\@lastname}{\@shortname}.}\\[15pt]%
	\fontsize{50}{50}\selectfont\@ifundefined{@shorttitle}{\@title}{\@shorttitle}.%
}


\def\firstname#1{\def\@firstname{#1}}
\def\lastname#1{\def\@lastname{#1}}
\def\shortname#1{\def\@shortname{#1}}
\def\shorttitle#1{\def\@shorttitle{#1}}
\def\namesuffix#1{\def\@namesuffix{#1}}
\def\instrumentation#1{\def\@instrumentation{#1}}
\def\parts#1{\def\@parts{#1}}

\firstname{\relax}
\lastname{\relax}
\namesuffix{\relax}
\instrumentation{\relax}
\parts{\relax}


\def\maketitle{%
\begin{titlepage}%
	\Large%
	{\@titlehead}%
	\vfill%
	{\strut\@firstname}\\%
	{\sbseries\color{oldred}\strut\@lastname}\\%
	{\strut\@namesuffix}%
	\vfill%
	{\sbseries\@title}\\%
	{\@subtitle}\\[\baselineskip]%
	{\itshape\@instrumentation}%
	\vfill%
	{\itshape\@parts}\hspace*{\fill}\raisebox{0pt}[0pt][0pt]{\includegraphics{ees_logo}}%
\end{titlepage}%
}


\newif\iftemplate\templatetrue
\newif\ifprintreport\printreportfalse
\directlua{
scores = {
  ob1 = "Oboe I",
  ob2 = "Oboe II",
  ottoni = "Clarino I, II in C\string\\newline Timpani in C–G",
  vl1 = "Violino I",
  vl2 = "Violino II",
  vla = "Viola",
  coro = "Coro",
  org = "Organo",
  b = "Bassi",
  full_score = "Full Score"
}

last_arg = arg[\string#arg]
texio.write("Last argument: " .. last_arg)
if not (scores[last_arg] == nil) then
  tex.print("\string\\def\string\\lypdfname{" .. last_arg .. "}")
  tex.print("\string\\parts{" .. scores[last_arg] .. "}")
  if (last_arg == "full_score") then
    tex.print("\string\\printreporttrue")
  end
end
}


\@ifundefined{lypdfname}{%
  \templatefalse
  \printreporttrue
  \parts{Draft}
}{\templatetrue}

\makeatother






\begin{document}
\frenchspacing

\titlehead{\fancytitlehead}
\firstname{Johann Heinrich}
\lastname{Rolle}
\title{Der leidende Jesus}
\subtitle{Passions-Cantate\\WacR III:7\\(D-B Mus.ms 18709/1)}
\instrumentation{S, A, T, B (solo), S, A, T, B (coro), 2 ob, 2 clno, timp, 2 vl, vla, b, org}
\maketitle


\thispagestyle{empty}

\vspace*{\fill}

\raisebox{-4mm}{\includegraphics{byncsaeu}}\hspace*{1em}Wolfgang Esser-Skala, 2020

© 2020 by Wolfgang Esser-Skala. This edition is licensed under the Creative Commons Attribution-NonCommercial-ShareAlike 4.0 International License. To view a copy of this license, visit \url{http://creativecommons.org/licenses/by-nc-sa/4.0/}. 

Music engraving by LilyPond 2.18.0 (\url{http://www.lilypond.org}).\\
Front matter typeset with Source Sans Pro and Fredericka the Great.

\textit{First version, October 2020}

\vspace*{2cm}

\ifprintreport
\chapter*{Critical Report.}

This edition bases upon a copy in the Königliche Bibliothek zu Berlin, which has been digitized by the Staatsbibliothek zu Berlin – Preußischer Kulturbesitz. The digital version of the manuscript is available at \url{http://resolver.staatsbibliothek-berlin.de/SBB000281F600000000} (siglum Mus.ms. 18709/1).

In general, this edition closely follows the manuscript. Any changes that were introduced by the editor are indicated by italic type (lyrics, dynamics and directions), parentheses (expressive marks and bass figures) or dashes (slurs and ties). Accidentals are used according to modern conventions. Asterisks denote changes that are clarified in the detailed remarks below.\footnote{Abbreviations: A, alto; B, bass; b, basses; clno, clarion; Ms, manuscript; ob, oboe; org, organ; r, rest; S, soprano; T,~tenor; timp, timpani; vl, violin; vla, viola.}

\bigskip

\begin{longtable}{lll L{10cm}}
	\toprule
	\itshape Mov. & \itshape Bar & \itshape Staff & \itshape Note \\
	\midrule \endhead
	1.3  & 20  & A   & 1st quarter in Ms: g′4 \\
	1.6  & 109 & org & bar in Ms: G2–r4 \\
	1.10 & 52  & T   & grace note missing in Ms \\
	1.11 & 25f & T   & bars in Ms: c′2–g2–a2–f′2 \\
	1.21 & –   & –   & instruments missing in Ms \\
	2.1  & 67  & org & 1st quarter in Ms: A4 \\
	\bottomrule
\end{longtable}

Bass figures appear in the following movements (bars in parentheses): 1.1, 1.3, 1.5 (1), 1.7, 1.8, 1.10 (94), 1.11, 1.12 (2, 18, 26), 1.14 (3, 6, 9), 1.16 (1, 2, 15, 17), 1.18 (11, 16, 17), 1.21, 2.1 (18, 20, 22, 59, 60, 79), 2.3 (5, 24), 2.4 (8, 16, 17, 20, 28). The remaining bass figures were added by the editor.

This edition has been compiled and checked with utmost diligence. Nevertheless, errors and mistakes cannot be totally excluded. Please report any errors and mistakes to \url{wolfgang@esser-skala.at} or create an issue or pull request on the edition’s GitHub page \url{https://github.com/skafdasschaf/rolle-der-leidende-jesus-WacR-III-7}. Your help will be greatly appreciated.

\bigskip
\textit{Salzburg, October 2020\\
Wolfgang Esser-Skala}

\cleardoublepage
\chapter*{Contents.}

\lyrefpart{erstertheil}

\lyrefsection{omeineseel}

\begin{description}
	\item[Coro]
	O, meine Seel, ermuntre dich,\\
	daß ganz auf Jesu Leiden sich\\
	voll Andacht Herz und Auge lenkt,\\
	und Himmelswonne dann dich tränkt.
	
	Anbetend folge nun ihm nach,\\
	von Hohn zu Hohn, von Schmach zu Schmach,\\
	bis zu dem Kreuz, daran er stirbt\\
	und Sündern Gottes Huld erwirbt.
	
	Dann sage: Jesu, du mein Heil,\\
	ich bin dein Eigenthum und Theil,\\
	wie du mein bist, so bleib ich dein,\\
	und will mich ganz dir ewig weyhn.
\end{description}

\lyrefsection{derherrneiget}

\begin{description}
	\item[Coro]
	Der Herr neiget den Himmel und fähret herab.\\
	Dunkel ist unter seinen Füßen,\\
	und in finstern Wolken ist er verborgen.\\\relax
	[Ps. 18:10, 12]
\end{description}

\lyrefsection{jerusalem}

\begin{description}
	\item[Alto]
	Jerusalem! Jerusalem!\\
	Welch eine Nacht senkt über dich\\
	vom Himmel sich hernieder!\\
	Kommt Gott, der zum Gericht erwacht,\\
	und fodert der Propheten Blut\\
	von deinen Händen wieder?\\
	Steigt er zum Weltgericht herab von seinem Thron,\\
	der Oelberg erbebet, es rauschet der Kidron,\\
	von fern her rauscht der Donnerton.\\
	Wo ist der größte der Propheten, dein Jesus?\\
	Fleh ihn an, dich zu verteten.\\
	Wo find ich ihn, den meine Seele liebt?\\
	Hör ich ihn dort betrübt\\
	aus jener Mitternacht her klagen?\\
	Er ists.\\
	Mit welchen dunkeln Plagen\\
	hat ihn sein Gott geschlagen!
\end{description}

\lyrefsection{gerechter}

\begin{description}
	\item[Alto]
	Gerechter! Welche Qual erschrecket\\
	den eingebeugten Heldenmuth!\\
	Du bist statt Schweiß mit Blut bedecket,\\
	und Thränen fließen in dein Blut.\\
	Ach, weint der Heiligste selbst Zähren,\\
	faßt Schrecken ihn und banger Schmerz,\\
	wer wird mir Sünder Trost gewähren,\\
	wer gießet Ruh in dieses Herz?\\
	Wie soll ich vor dem Richter stehn,\\
	vor dem mein Jesus will vergehn?
\end{description}

\lyrefsection{labyrinth}

\begin{description}
	\item[Basso]
	O Labyrinth! der der Natur gebot,\\
	vor dem der Sturmwind schwieg,\\
	der dir, o Tod, allmächtig, seinen Raub entrißen,\\
	liegt hier, von Finsternissen\\
	umschattet und von Angst gedrängt,\\
	und leidet, was die Seele nicht gedenkt,\\
	und jammert Todestöne.\\
	Ach! wer erhellt mir diese dunkle Sonne?\\
	Mein Jesus, meiner Seelen Freund,\\
	was ist es, daß dein Auge weint?\\
	Heb aus dem Staube dich.\\
	Ein Blick von dir, nur einer, lehre mich,\\
	dies Todesjammern, dies Zagen\\
	verstehn und zu ertragen.\\
	Er höret mich: sanft ist sein Blick\\
	und ruft das Leben mir zurück.\\
	Sein Auge spricht: Die Arbeit dieser Nacht\\
	hat mir der Menschen Schuld gemacht.
\end{description}

\lyrefsection{singtihr}

\begin{description}
	\item[Coro]
	Singt, ihr Himmel, Gott ist Liebe,\\
	wunderbarlich ist sein Rath.\\
	Sing, o Erde, Gott ist Liebe,\\
	bey der Thaten größten That.\\
	Ihr Geschlechter der Erlösten,\\
	dem, der leidet, uns zu trösten,\\
	laßt uns ewig dankbar seyn.\\
	Jedes Herz, das ihn verkennet,\\
	nicht für ihn und Tugend brennet,\\
	faß einst ewig diese Pein.
\end{description}

\lyrefsection{liebedie}

\begin{description}
	\item[Coro]
	Liebe, die du mich zum Bilde\\
	deiner Gottheit hast gemacht,\\
	Liebe, die du mich so milde\\
	nach dem Fall mit Heil bedacht,\\
	Liebe, dir ergeb ich mich,\\
	dein zu bleiben ewiglich.
\end{description}

\lyrefsection{nochherrscht}

\begin{description}
	\item[Tenore]
	Noch herrscht um ihn ein schauervolles Schweigen.\\
	Kein Laut ertönt.\\
	Wo seid ihr, Zeugen der Wunder, die er that?\\
	Wo seid ihr? Seid ihr entflohn?\\
	Vergaß ein jeder schon den Schwur,\\
	sein Leben hin für den Göttlichen zu geben?\\
	Wie hatt’ er euch so lieb!\\
	Itzt wendet sich der Liebende\\
	und suchet, Simon, dich, und euch, Zebedäiden.\\
	Mit diesem Trost, nur euch zu sehn, zufrieden,\\
	naht er sich euch.\\
	Sie schlummern, und er spricht:\\
	Ach Simon, du vermagst es nicht,\\
	auch eine Stunde nur mit mir zu wachen!\\
	Zwar willig ist der Geist,\\
	doch drückt den Schwachen das Fleisch zur Erd herab!\\
	Ach, wacht und betet, daß ihr singt\\
	und nicht der Anfechtung erliegt.
\end{description}

\lyrefsection{wachetstehet}

\begin{description}
	\item[Coro]
	Wachet, stehet im Glauben,\\
	seid männlich und seid stark.\\\relax
	[1 Cor 16:13]
\end{description}

\lyrefsection{herrermuntre}

\begin{description}
	\item[Tenore]
	Herr, ermuntre du uns Schwachen,\\
	daß wir wachen,\\
	daß der Geist sich zu dir schwingt,\\
	kämpft und ringt.\\
	Zeig unsrem Glauben jene Höhe,\\
	wo wir gekrönt einst vor dir stehn,\\
	wenn wir hier wachend dich gefunden,\\
	dir nachgekämpft und überwunden,\\
	und wo der Engel Lied den singt,\\
	der wacht und ringt.
\end{description}

\lyrefsection{wieist}

\begin{description}
	\item[Coro]
	Wie ist der Mensch so schwach!\\
	So viel sein Muth versprach,\\
	doch liegt er da im Staube.\\
	Erloschen ist sein Glaube,\\
	sein kühner Muth gedämpfet.\\
	Ach, Christus wacht und kämpfet.
\end{description}

\lyrefsection{docherverlaesst}

\begin{description}
	\item[Basso]
	Doch er verläßt die Schlummernden\\
	und eilt aufs neu, in das Gericht zu gehn,\\
	denn immer noch, ich staune,\\
	tönt ihm in seinem Ohr\\
	des Weltgerichts Posaune.\\
	Noch immer richtet der Richter der Welt den,\\
	der als Mittler zwischen Gott\\
	und das Geschlecht der Menschen sich stellt.\\
	Empfindungen, stark wie der Tod, erschüttern ihn.\\
	Die Erd erbebet wieder,\\
	die Nacht hängt schrecklicher vom Oelberg nieder.\\
	Und du, der du in Gottes Unschuld prangst,\\
	ringst mit des ewgen Todes Angst.\\
	Doch schau, ein Blitz zertheilet die Nacht,\\
	ein Seraph eilet herab und singt ein Lied\\
	von deinem Vater dir,\\
	der huldvoll auf dich sieht,\\
	und stärket dich.
\end{description}

\lyrefsection{werdurchschaut}

\begin{description}
	\item[Basso]
	Wer durchschaut, wie wunderbar\\
	Gott ist in seinen Werken!\\
	Ach! ein Engel muß sogar\\
	den Herrn der Welten stärken.\\
	Wenn ich einst von hinnen scheide,\\
	singen Engel mir zur Ruh.\\
	Nun eilst du zu unsrer Freude\\
	Gottes Vaterarmen zu.
\end{description}

\lyrefsection{gestaerkterhebt}

\begin{description}
	\item[Alto]
	Gestärkt erhebt mein Jesus sich\\
	und geht der Schaar entgegen,\\
	die ihn voll Mordlust sucht,\\
	durch dich, Ischarioth geführt.\\
	Bekümmert seiner Freunde wegen\\
	spricht er: Wen suchet ihr?\\
	und Allmacht ist sein Blick.\\
	Schnell stürzt die Schaar zurück\\
	und sinkt und liegt,\\
	wie auf dem Schlachtfeld Todte liegen.\\
	Doch die Betäubung weicht,\\
	sie schauen voll Vergnügen\\
	bey ihrer Fackeln Schein\\
	nach dem Verräther hin,\\
	der tritt zu ihm und küßet ihn.\\
	Und Jesus blickt voll Mitleid,\\
	voller Ruh auf ihn, und sagt:\\
	Juda, verräthest du des Menschen Sohn\\
	mit einem Kuß?\\
	So sanft sucht er ihn,\\
	an des Abgrunds Stufen,\\
	zur Reu zurück zu rufen.
\end{description}

\lyrefsection{gottmitblicken}

\begin{description}
	\item[Alto]
	Gott, mit Blicken deiner Gnade,\\
	hilf, daß ich vom Lasterpfade\\
	bald den Fuß zurücke zieh,\\
	meines Vaters Stimme höre,\\
	wieder reuend zu ihm kehre\\
	und des Abgrunds Tiefe flieh.\\
	Dann nimm den Reuenden\\
	mit Vaterblicken an.
\end{description}

\lyrefsection{siebinden}

\begin{description}
	\item[Tenore]
	Sie binden ihn; er reicht der Schaar\\
	die Hände willig dar,\\
	indeß die kleine Zahl der Freunde sich zerstreuet.\\
	Mit Höllenfreuden freuet sich Kaiphas,\\
	setzt auf den Richtstuhl sich,\\
	und richtet dich, der du voll Ruh\\
	den Blick zum Himmel lenkest,\\
	nur Gott, die Ewigkeit und die Erlösten denkest.\\
	Der Heiligste steht in der Sünder Gericht,\\
	hört die erkauften Zeugen nicht,\\
	der Lästerung, des Spottes Stimme nicht,\\
	und schweigt.
\end{description}

\lyrefsection{lammdas}

\begin{description}
	\item[Coro]
	Lamm, das von verruchten Zungen\\
	frech verhöhnet dennoch schwieg!\\
	Stiller Muth bey Lästerungen,\\
	welch ein edelmüthger Sieg!\\
	Muß ich gleichen Grimm empfinden,\\
	lehre mich gelaßen seyn,\\
	und will sich mein Zorn entzünden,\\
	flöß mir deine Sanftmuth ein.
\end{description}

\lyrefsection{dochkaiphas}

\begin{description}
	\item[Soprano]
	Doch Kaiphas, ergrimmt durch dieses Schweigen,\\
	reißt wütend sich hervor,\\
	und schon glüht ihm die Wang:\\
	Schweigst du zu dem, was diese zeugen? ruft er:\\
	Sprich, bey dem Ewigen beschwör ich dich,\\
	sprich, bist du Gottes Sohn?\\
	Und Jesus würdigt ihn, zu sagen: Ich bins.\\
	Von diesen Tagen, von nun an wirds geschehn,\\
	daß ihr des Menschen Sohn zur Rechten Gottes sehn\\
	und kommend in den Wolken werdet sehen,\\
	wenn er daher wird zum Gerichte gehen.
\end{description}

\lyrefsection{meinistdie}

\begin{description}
	\item[Soprano]
	Mein ist die Unsterblichkeit,\\
	die Unsterblichkeit ist mein.\\
	Jauchze deinem Leben, Seele,\\
	Gott wird Ewigkeit dir zur Dauer geben.\\
	Wenn euch wird das nahe Grab erschrecken,\\
	Todesbläße eure Wangen decken,\\
	wenn einst diese Hütte sinkt,\\
	schaut nach jenen Wolken dann, ihr Frommen,\\
	auf denselben wird der Richter kommen,\\
	wenn er euch ins Leben winkt.
\end{description}

\lyrefsection{christushatdem}

\begin{description}
	\item[Coro]
	Christus hat dem Tode die Macht genommen\\
	und das Leben und ein unvergängliches Wesen\\
	ans Licht gebracht.\\\relax
	[2 Tim 1:10]
\end{description}

\lyrefsection{wenndort}

\begin{description}
	\item[Coro]
	Wenn dort, Herr Jesu, wird vor deinem Throne\\
	auf meinem Haupte stehn die Ehrenkrone,\\
	da will ich dir, wenn alles wird wohlklingen,\\
	Lob und Dank singen.
\end{description}

\lyrefpart{zweytertheil}

\lyrefsection{weristder}

\begin{description}
	\item[Alto]
	Wer ist der Mann, der unter jenen Knechten\\
	der Grausamkeit entschloßen steht,\\
	mit ihnen eifrig scheint zu rechten?\\
	Sie rufen: Seht! auch dieser war ein Galiläer,\\
	wir sahn ihn bey dem Nazaräer.\\
	Bist du es, Simon?\\
	Ach, du wirst ihn nicht verkennen, den Göttlichen;\\
	nein, so unedel bist du nicht!\\
	O weh! er schwört und spricht:\\
	Ich kenne diesen Menschen nicht.\\
	Und Jesus blickt ihn an,\\
	voll Ruh, voll Ernst und Schmerzen,\\
	ein Dolch dringt mit dem Blick zu Petri Herzen.\\
	Er wendet sich, und geht hinaus,\\
	und weinet bitterlich und klagt:\\
	Mein Freund, mein Freund!\\
	Ach, was that ich?\\
	Geliebt, gewarnt von dir,\\
	verläugnet dich dein Simon.\\
	Tödtend drang sein Blick in mein Gebein.\\
	Ich fühl, ich fühle Todespein,\\
	du Göttlicher, wirst nun mich auch nicht kennen,\\
	vor deinem Vater nicht, vor Engeln mich nicht nennen.\\
	Ja, nenne mich nur nicht, ich habs verdient,\\
	verstoß mich im Gericht!\\
	Rauscht, ihr Schrecken dieser Nacht,\\
	rauschet mir Tod und Verderben.\\
	Fluch hab ich auf mein Haupt gebracht,\\
	ach, könnt ich sterben!\\
	Mein Vater, dieses Herz erbebt,\\
	dies Auge weint, erbarm, o Vater,\\
	erbarm dich meiner.\\
	Viel sündigen an ihm,\\
	der Reue Pfeil fühl keiner wie ich,\\
	er gräbet Tod mir ein!\\
	Ach laß mich, eh er stirbt,\\
	laß mich ihn sehen,\\
	von ihm Verzeihung zu erflehen.\\
	Dann, wenn er sterbend mir verzeiht,\\
	dann soll, so lang der Herr zu leben mir gebeut,\\
	vor allen Menschen dieser Mund ihn nennen,\\
	ihn für den theuren, besten, göttlichsten erkennen,\\
	dann wein ich auf sein Grab.\\
	So jammert er, und fühlt der heißen Reue Pein,\\
	Gott, Mittler, ach, erbarme du dich sein.
\end{description}

\lyrefsection{gottdudonnerst}

\begin{description}
	\item[Basso]
	Gott, du donnerst zu den Sündern\\
	deinen Fluch vom Richterstuhl\\
	bis hinab zum Feuerpfuhl.\\
	Ach, wenn meine Zähren fließen,\\
	wenn die Reu mein Herz zerrißen,\\
	wenn das strafende Gewissen\\
	auf euch gießet Höllenpein,\\
	ach, Herr, höre dann mein Schreyn,\\
	gib mir Trost, die Angst zu lindern.
\end{description}

\lyrefsection{achseele}

\begin{description}
	\item[Coro]
	Ach Seele, schau um welchen Preis\\
	dein Heiland dich erkaufet.\\
	Für dich rang er im Todesschweiß,\\
	für dich im Blut getaufet.\\
	Ach Seele, sorge, daß dich nie\\
	die Sünd in ihre Netze zieh,\\
	o Menschenfurcht erschüttre.\\
	Reizt dich die Welt, ach höre nicht,\\
	schau hin ins furchtbare Gericht,\\
	das Jesum traf, und zittre.
\end{description}

\lyrefsection{dertagbricht}

\begin{description}
	\item[Basso]
	Der Tag bricht an, der festliche,\\
	der große Tag, gesendet von der Liebe.\\
	Entflammet von dem Triebe der Rach und Wuth,\\
	versammelte der Priester Haufe sich.\\
	Erfüllet mir Verderben rief er:\\
	Auf Golgatha, am Kreuze soll er sterben.\\
	Triumphvoll reißen sie ihn zu Pilatus Richtstuhl fort.\\
	Den feigen Römer schreckt der Kläger Wort\\
	und ihr Geschrey, davon das Richthaus bebet:\\
	Wenn durch dein Mitleid dieser Jesus lebet,\\
	bist du des Kaisers Feind.\\
	Ihm sinkt der Muth, doch ruft er noch:\\
	Ich bin unschuldig an dem Blut des Frommen.\\
	Da ertönt das Schreckenswort der Sünder:\\
	Über uns komme sein Blut und über unsre Kinder.
\end{description}



\cleardoublepage
\fi

\iftemplate
\includepdf[pages=-]{../out/\lypdfname.pdf}
\fi

\end{document}